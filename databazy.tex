\documentclass[10pt,a4paper]{article}
\usepackage[slovak]{babel}
\usepackage[utf8]{inputenc}
\usepackage{amsmath}
\usepackage{amsfonts}
\usepackage{amssymb}
\usepackage[unicode]{hyperref}
\usepackage{graphicx}

\textwidth 6.5in
\oddsidemargin 0.0in
\evensidemargin 0.0in

\title{Poznámky z Úvodu do databázových systémov - materiál na štátnice}
\date{16.06.2012}
\author{Peter Csiba, petherz@gmail.com, \url{https://github.com/Petrzlen/fmfi-poznamky}} 

\begin{document}
\maketitle
\tableofcontents

\clearpage

%%%%%%%%%%%%%%%%%%%%%%%%%%%%%%%%%%%%%%%%%%%%%%%%%%%%%%%%%%%%%%%%%%%%%%%%%%%%%%%%
\section*{Úvod}   

Text je poznámkami k oficiálnym \href{http://new.dcs.fmph.uniba.sk/index.php/Studium/Bakalarske/StatneSkusky}{štátnicovým otázkam} a boli spísané počas učenia sa na ne.
Poznámky sa nesnažia ísť do hĺbky (na to je H.Garcia-Molina, Ullman a Wikipédia).
Naopak, snažia sa priniesť intuitívnu predstavu o algoritmoch a pojmoch a dávať ich do súvisu.

Poznámky sú organizované podľa štátnicových otázok, snažia sa minimalizovať omáčku a nevysvetlujú a neuvádzajú do problematiky. Text je určený čitateľom, ktorý sa už s hlavnými pojmami stretli.
Autor považuje všetky otázky za rovnako dôležité a odporúča v prípade slabšieho pochopenia samostatne vypracovať niektoré staré \href{http://www.dcs.fmph.uniba.sk/~plachetk/TEACHING/DB2011/index.html}{písomky}.

Autor absolvoval základný test predmetu Úvodu do databázových systémov s hodnotením A, a má základné skúsenosti s administráciou a návrhom databáz. Napriek tomu autor \underline{neručí za kvalitu a úplnosť textu} a čitateľov \underline{aj} preto autor \underline{prudko odporúča pozrieť si aj iné zdroje}. Uvedieme citát\footnote{Dr. Tomas Plachetka, Uvod do databazovych systemov 2011/2012 Zima} ''Tieto slajdy sú sprievodcom pri prednáške, nie sú myslené ako náhrada
prednášky či nebodaj knihy. K príprave na skúšku nestačí len prečítať slajdy'', niečo také sú aj tieto poznámky.  

Nakoniec poznamenajme, že autor sa snažil písať pravdu a len pravdu, keďže jeho odpoveď na záverečných skúškach vychádza z tototo materiálu.
Ak čitateľ chce prispieť ku kvalite textu, nech autorovi napíše a ten mu udelí prístup do repozitára.

P.S. Autor zistil, že názvoslovie pijan, ľúbi, krčma, alkohol použil už napríklad \href{http://csip.sk/uploads/ullman.pdf}{Ullman}.

    
%%%%%%%%%%%%%%%%%%%%%%%%%%%%%%%%%%%%%%%%%%%%%%%%%%%%%%%%%%%%%%%%%%%%%%%%%%%%%%%%
%%%%%%%%%%%%%%%%%%%%%%%%%%%%%%%%%%%%%%%%%%%%%%%%%%%%%%%%%%%%%%%%%%%%%%%%%%%%%%%%
\section{Dátové modely}
%%%%%%%%%%%%%%%%%%%%%%%%%%%%%%%%%%%%%%%
\subsection{Trojschémová architektúra (ANSI sparc)}

%%%%%%%%%%%%%%%%%%%%%%%%%%%%%%%%%%%%%%%
\subsection{Entitno-relačný model}

%%%%%%%%%%%%%%%%%%%%%%%%%%%%%%%%%%%%%%%
\subsection{Relačný model, relačná algebra}

%%%%%%%%%%%%%%%%%%%%%%%%%%%%%%%%%%%%%%%
\subsection{Negácia a rekurzia v relačnej algebre}

%%%%%%%%%%%%%%%%%%%%%%%%%%%%%%%%%%%%%%%
\subsection{Súvis relačnej algebry s inými dotazovacími jazykmi}
    
%%%%%%%%%%%%%%%%%%%%%%%%%%%%%%%%%%%%%%%%%%%%%%%%%%%%%%%%%%%%%%%%%%%%%%%%%%%%%%%%
%%%%%%%%%%%%%%%%%%%%%%%%%%%%%%%%%%%%%%%%%%%%%%%%%%%%%%%%%%%%%%%%%%%%%%%%%%%%%%%%
\section{Relačný kalkul}

%%%%%%%%%%%%%%%%%%%%%%%%%%%%%%%%%%%%%%%
\subsection{Predikátová interpretácia relačnej algebry}

%%%%%%%%%%%%%%%%%%%%%%%%%%%%%%%%%%%%%%%
\subsection{Negácia, doménovo nezávislé a bezpečné formuly}

%%%%%%%%%%%%%%%%%%%%%%%%%%%%%%%%%%%%%%%
\subsection{Relačný kalkul (doménový)}

%%%%%%%%%%%%%%%%%%%%%%%%%%%%%%%%%%%%%%%
\subsection{Súvis relačného kalkulu s inými dotazovacími jazykmi}
    
%%%%%%%%%%%%%%%%%%%%%%%%%%%%%%%%%%%%%%%%%%%%%%%%%%%%%%%%%%%%%%%%%%%%%%%%%%%%%%%%
%%%%%%%%%%%%%%%%%%%%%%%%%%%%%%%%%%%%%%%%%%%%%%%%%%%%%%%%%%%%%%%%%%%%%%%%%%%%%%%%
\section{Datalog}

%%%%%%%%%%%%%%%%%%%%%%%%%%%%%%%%%%%%%%%
\subsection{Syntax a sémantika Datalogových programov}

%%%%%%%%%%%%%%%%%%%%%%%%%%%%%%%%%%%%%%%
\subsection{Súvis s relačným kalkulom}

%%%%%%%%%%%%%%%%%%%%%%%%%%%%%%%%%%%%%%%
\subsection{Výpočet dotazu na Datalogový program}

%%%%%%%%%%%%%%%%%%%%%%%%%%%%%%%%%%%%%%%
\subsection{Negácia}

%%%%%%%%%%%%%%%%%%%%%%%%%%%%%%%%%%%%%%%
\subsection{Bezpečnosť Datalogových programov}
    
%%%%%%%%%%%%%%%%%%%%%%%%%%%%%%%%%%%%%%%%%%%%%%%%%%%%%%%%%%%%%%%%%%%%%%%%%%%%%%%%
%%%%%%%%%%%%%%%%%%%%%%%%%%%%%%%%%%%%%%%%%%%%%%%%%%%%%%%%%%%%%%%%%%%%%%%%%%%%%%%%
\section{Relačná algebra}

%%%%%%%%%%%%%%%%%%%%%%%%%%%%%%%%%%%%%%%
\subsection{Operátory relačnej algebry}

%%%%%%%%%%%%%%%%%%%%%%%%%%%%%%%%%%%%%%%
\subsection{Multimnožinová interpretácia relácií}

%%%%%%%%%%%%%%%%%%%%%%%%%%%%%%%%%%%%%%%
\subsection{Grupovanie a agregácia}

%%%%%%%%%%%%%%%%%%%%%%%%%%%%%%%%%%%%%%%
\subsection{Rekurzia, výpočet pevného bodu}

%%%%%%%%%%%%%%%%%%%%%%%%%%%%%%%%%%%%%%%
\subsection{Súvis relačnej algebry s inými dotazovacími jazykmi}
    
%%%%%%%%%%%%%%%%%%%%%%%%%%%%%%%%%%%%%%%%%%%%%%%%%%%%%%%%%%%%%%%%%%%%%%%%%%%%%%%%
%%%%%%%%%%%%%%%%%%%%%%%%%%%%%%%%%%%%%%%%%%%%%%%%%%%%%%%%%%%%%%%%%%%%%%%%%%%%%%%%
\section{Jazyk SQL}

%%%%%%%%%%%%%%%%%%%%%%%%%%%%%%%%%%%%%%%
\subsection{Programovanie v SQL (Data Definition Language, Data Manipulation Language)}

%%%%%%%%%%%%%%%%%%%%%%%%%%%%%%%%%%%%%%%
\subsection{Negácia a rekurzia v SQL}

%%%%%%%%%%%%%%%%%%%%%%%%%%%%%%%%%%%%%%%
\subsection{Súvis SQL s inými dotazovacími jazykmi}
    
%%%%%%%%%%%%%%%%%%%%%%%%%%%%%%%%%%%%%%%%%%%%%%%%%%%%%%%%%%%%%%%%%%%%%%%%%%%%%%%%
%%%%%%%%%%%%%%%%%%%%%%%%%%%%%%%%%%%%%%%%%%%%%%%%%%%%%%%%%%%%%%%%%%%%%%%%%%%%%%%%
\section{Teória navrhovania relačných báz dát}

%%%%%%%%%%%%%%%%%%%%%%%%%%%%%%%%%%%%%%%
\subsection{Funkčné závislosti, Armstrongove axiómy, uzáver množiny atribútov, uzáver množiny funkčných závislostí}

%%%%%%%%%%%%%%%%%%%%%%%%%%%%%%%%%%%%%%%
\subsection{Pokrytie a minimálne pokrytie množiny funkčných závislostí}

%%%%%%%%%%%%%%%%%%%%%%%%%%%%%%%%%%%%%%%
\subsection{Nadkľúče a kľúče}
    
%%%%%%%%%%%%%%%%%%%%%%%%%%%%%%%%%%%%%%%%%%%%%%%%%%%%%%%%%%%%%%%%%%%%%%%%%%%%%%%%
%%%%%%%%%%%%%%%%%%%%%%%%%%%%%%%%%%%%%%%%%%%%%%%%%%%%%%%%%%%%%%%%%%%%%%%%%%%%%%%%
\section{Normálne formy}

%%%%%%%%%%%%%%%%%%%%%%%%%%%%%%%%%%%%%%%
\subsection{3NF, BCNF}

%%%%%%%%%%%%%%%%%%%%%%%%%%%%%%%%%%%%%%%
\subsection{Algoritmy pre dekompozíciu do normálnych foriem}

%%%%%%%%%%%%%%%%%%%%%%%%%%%%%%%%%%%%%%%
\subsection{Bezstratovosť dekompozície}
    
%%%%%%%%%%%%%%%%%%%%%%%%%%%%%%%%%%%%%%%%%%%%%%%%%%%%%%%%%%%%%%%%%%%%%%%%%%%%%%%%
%%%%%%%%%%%%%%%%%%%%%%%%%%%%%%%%%%%%%%%%%%%%%%%%%%%%%%%%%%%%%%%%%%%%%%%%%%%%%%%%
\section{Transakcie}

%%%%%%%%%%%%%%%%%%%%%%%%%%%%%%%%%%%%%%%
\subsection{Požiadavky na transakčný systém (ACID)}

%%%%%%%%%%%%%%%%%%%%%%%%%%%%%%%%%%%%%%%
\subsection{Architektúra transakčného systému}

%%%%%%%%%%%%%%%%%%%%%%%%%%%%%%%%%%%%%%%
\subsection{Rozvrhy}

%%%%%%%%%%%%%%%%%%%%%%%%%%%%%%%%%%%%%%%
\subsection{Triedy sériovateľnosti a obnoviteľnosti}
    
%%%%%%%%%%%%%%%%%%%%%%%%%%%%%%%%%%%%%%%%%%%%%%%%%%%%%%%%%%%%%%%%%%%%%%%%%%%%%%%%
%%%%%%%%%%%%%%%%%%%%%%%%%%%%%%%%%%%%%%%%%%%%%%%%%%%%%%%%%%%%%%%%%%%%%%%%%%%%%%%%
\section{Implementácia sériovateľnosti a obnoviteľnosti v transakčných systémoch}
%%%%%%%%%%%%%%%%%%%%%%%%%%%%%%%%%%%%%%%
\subsection{Testy sériovateľnosti}

%%%%%%%%%%%%%%%%%%%%%%%%%%%%%%%%%%%%%%%
\subsection{Algoritmy izolácie, zámky, časové pečiatky, validácia}

%%%%%%%%%%%%%%%%%%%%%%%%%%%%%%%%%%%%%%%
\subsection{Uviaznutie (deadlock) a metódy riešenia uviaznutia}

%%%%%%%%%%%%%%%%%%%%%%%%%%%%%%%%%%%%%%%
\subsection{Algoritmy onbovy, log-file, checkpointing, backup}
    
%%%%%%%%%%%%%%%%%%%%%%%%%%%%%%%%%%%%%%%%%%%%%%%%%%%%%%%%%%%%%%%%%%%%%%%%%%%%%%%%
%%%%%%%%%%%%%%%%%%%%%%%%%%%%%%%%%%%%%%%%%%%%%%%%%%%%%%%%%%%%%%%%%%%%%%%%%%%%%%%%
\section{Fyzická organizácia}

%%%%%%%%%%%%%%%%%%%%%%%%%%%%%%%%%%%%%%%
\subsection{Dvojúrovňový model pamäti a organizácie dát}

%%%%%%%%%%%%%%%%%%%%%%%%%%%%%%%%%%%%%%%
\subsection{Indexové stromy, hashovanie}

%%%%%%%%%%%%%%%%%%%%%%%%%%%%%%%%%%%%%%%
\subsection{Operátory fyzickej algebry}

%%%%%%%%%%%%%%%%%%%%%%%%%%%%%%%%%%%%%%%
\subsection{Implementácia vybraných fyzických operátorov (merge-sort, nested-loop join)}


%%%%%%%%%%%%%%%%%%%%%%%%%%%%%%%%%%%%%%%%%%%%%%%%%%%%%%%%%%%%%%%%%%%%%%%%%%%%%%%%
%%%%%%%%%%%%%%%%%%%%%%%%%%%%%%%%%%%%%%%%%%%%%%%%%%%%%%%%%%%%%%%%%%%%%%%%%%%%%%%%
\section*{Referencie a odporúčaná literatúra}
\begin{itemize}                                
\item \href{http://www.dcs.fmph.uniba.sk/~plachetk/TEACHING/DB2011/index.html}{Úvodu do databázových systémov - Plachetka}.        
\item \href{http://www.dcs.fmph.uniba.sk/~sturc/databazy/uvod/}{Úvodu do databázových systémov - Šturc} - v niečom detailnejší.        
\item \href{http://infolab.stanford.edu/~widom/cs145/}{Úvodu do databázových systémov - Stanford University} - základ podobný.


\item \href{http://csip.sk/uploads/plachetka\_uvod\_do\_databaz\_2011.pdf}{Plachetkove slide-i}.
\item \href{http://csip.sk/uploads/ullman.pdf}{Ullmanove slide-i}.
\item \href{http://fmfi-uk.hq.sk/Informatika/Uvod\%20Do\%20Databazovych\%20Systemov/prednasky/}{Mandos}.
\item H. Garcia-Molina, J.D. Ullman, J. Widom: Database Systems, The Complete Book, Prentice Hall, 2003
\item R. Elmasri, S.B. Navathe: Fundamentals of Database Systems, Addison-Wesley, 2006
\item Na poslednú otázku\footnote{
Na ktorej už zopár ľudí dostalo Fx.
}: S. Lightstone, T.J. Teorey, T. Nadeau: Physical Database Design, Morgan Kaufmann, 2007
\end{itemize}

\end{document}
