\documentclass[12pt,a4paper]{article}

\usepackage{listings}
\usepackage{graphicx}
\usepackage{tabularx} 
\usepackage{hyperref} 

\lstset{
language=sh
,breaklines=true
,basicstyle=\ttfamily
, showstringspaces=false}

\author{Peter Csiba}
\textwidth 6.5in
\oddsidemargin 0.0in
\evensidemargin 0.0in

\title{Database systems 2 - Final}
\date{09-May-2014}
\author{Peter Csiba, petherz@gmail.com}

\begin{document}
\maketitle

\section{Introduction}
Keyword badly defined + "ullman" or \url{cseweb.ucsd.edu/classes/sp14/.../} leads to explanation. 

\begin{itemize} 
\item \url{http:// infolab.stanford.edu/~ullman/.../slides01-14.pdf} 
\item \url{http://infolab.stanford.edu/~ullman/cs345-notes.html}
\item \url{http://www.dcs.fmph.uniba.sk/~plachetk/TEACHING/RLDB2014/index.html} 
\end{itemize} 

\section{Uvodna prednaska}

\subsection{Funkcne symboly v Datalogu, suvis s XML}
\begin{itemize} 
\item $n$-ary function returns any type. Nullary function = constant. 
\item \emph{Characteristic predicate} $P(x_1,\ldots,x_n, Y) \equiv \left(f(x_1, \ldots, x_n) = Y\right)$
\item \emph{Functor} := functional symbol in datalog. 
\end{itemize} 

\subsection{Termy, Herbrandove univerzum, substitucie, matching, unifikacia}
\begin{itemize} 
\item \emph{Term} je premenná, alebo funkcia zložená z termov a konštánt. Term je \emph{základný} ak neobsahuje premenné. Terms as data structures a(p(s(Baker\_Street), nr(221B)), c(London, pc(NW1\_6XE))). 
\item \emph{Herbrandove univerzum} := všetky termy potrebné na výpočet daného programu. 
\item \emph{Substitúcia} := $\tau = [\{X_i \Rightarrow t_i\}]$. Substitúcia prebieha paralelne a zapisuje sa $s = t \tau$. 
\item \emph{Usporiadanie termov} $t \sqsupseteq s \equiv \exists \Tau : s = t\Tau$. Minimálne termy sú konštanty a maximálne premenné. Dva termy sú rovnaké ak sa líšia len v pomenovaní premenných.
\item \emph{Usporiadanie substitúcií} - substitúcia $\tau \sqsubseteq \sigma$ ak $\exists \phi : \sigma = \tau \circ \phi$. Dve substitúcie sú \emph{ekvivalentné} ak ide o permutácie premenných. Ak nie sú ekvivalentné, tak $\sigma$ je \emph{špecializáciou} $\tau$. 
\item \emph{Term matching} - je daný ľubovoľný term $t$ a základný term $s$. Existuje $\tau : s = t\tau$? Priamočiarim riešením je \emph{recursive descent}. 
\item \emph{Unifikácia} (viacero variant, mi používame túto). Nájdite najvšeobecnejšiu substitúciu $t\sigma = s\sigma$. Spravíme si stromy pre oba termy $s$ a $t$, korene sú ekvivalentné, synovia ekvivaletných termov sú v poradí ekvivalentné. Urobíme reflexívno-tranzitívny uzáver a vzniknutá relácia ekvivalencie je hľadaná substitúcia. Poznáma: do premennej môže byť priradený ľubovoľný term. Dá sa to zrýchliť (Prívara, Ruzicka 1989). 
\end{itemize} 

\subsection{Preklad Datalogu s funkcnymi symbolmi do relacnej algebry}


\subsection{Operatory atov a vtoa}
Datalog rule \emph{head}s can accept any terms. Therefore when translating to relational algebra value sets satisfying these terms must be selected. 

\subsection{Naivna a seminaivna iteracia programov bez negacie}
Rekurzivne predikaty sa vyhodnocuju iterativne, kym je nejaka zmena. Tento porces skonci kvoli Tarskeho vete o pevnom bode. Semi-naivna pocita iba s rozdielmy a tak setri cas. Zalezi na poradi. 

\section{Negacia v databazach a logickom programovani}

\subsection{Logika, princip vylucenia tretieho, protirecenia}

\subsection{Formalne logicke systemy: propozicionalny kalkul, predikatovy kalkul, ...}

\subsection{Teorie a modely}

\subsection{Domenovo nezavisle formuly}

\subsection{Vypocet najmensieho modelu programu bez negacie (semi-) naivnou evaluaciou}

\subsection{Tarskeho veta o pevnom bode}

\subsection{Problem naivnej evaluacie programov s negaciou: minimalny pevny bod nemusi existovat}

TODO: Stratifikovany pevny bod
Stratified database: Nema rekurziu s negaciou \url{cseweb.ucsd.edu/classes/sp14/.../Datalog-neg09.pdf}.
(viac v K.R. Apt, H.A. Blair, and A. Walker, ‘‘Towards a theory of declarative knowledge,’’ in Foundations of Deductive Databases and Logic Programming, J. Minker, Ed., Morgan Kaufman, Los Altos, CA, 1988, pp. 89–148.)

\subsection{Stratifikovane a lokalne stratifikovane programy}
TODO: WAT? 

\subsection{Stabilne modely}

\subsection{Stabilne modely, Gelfond-Lifschitzova transformacia}

\subsection{Trojhodnotove modely, well-founded model}
Unfounded sets: \url{www.cs.uni-potsdam.de/wv/.../unfounded-anim.pdf} 
T.j. pravidla, ktore su bud SPOLU nepravdive, alebo nedokazatelne. Inymi slovami, ak je pravidlo v unfounded set, tak za predpokladu nepravdivosti ostatnych pravidiel v unfounded set a vsetkych implikacii v teorii je toto pravidlo nepravdive / nedokzatelne. 

\emph{Partial model} <T,F> je mnozina true, false. 

\subsection{Metody vypoctu well-founded modelu: maximal unfounded set, alternujuci pevny bod}

\subsection{Hierarchia semantik, priklady inych semantik}

\subsection{Rozsirenie relacnej algebry o semijoin a antijoin}

\subsection{Vstupne mnoziny, pseudokluce, bezpecnost programov so zabudovanymi (napr. aritmetickymi) predikatmi}
 
\emph{Safe Datalog rule}: A Datalog rule $p :- q_1, ..., q_n.$ is safe
\begin{itemize} 
\item if every variable that occurs in a negated subgoal also
appears in a positive subgoal and
\item if every variable that appears in the head of the rule also
appears in the body of the rule.
\end{itemize} 


\subsection{Vypocet programov s negaciou zdola nahor (Datalog) a zhora nadol (Prolog, SLD rezolucia)}

\section{Vypocet logickych programov}

\subsection{Datalog versus Prolog: vypocet zdola nahor vs zhora nadol}

\subsection{Rule-Goal Tree (RGT)}

\subsection{Sirenie vazieb a vysledkov: magicke a pomocne predikaty}

\subsection{Vypocet RGT do hlbky: expand\_goal, expand\_rule}

\subsection{Vypocet RGT do sirky: Queue-based RGT (QRGT)}

\subsection{Ozdoby predikatov a pravidiel, Rule-Goal Graph (RGG)}

\subsection{Usporiadanie podcielov v pravidle, zuniformnenie ozdob}

\subsection{Rektifikacia programu}

\subsection{Zabudovane podciele, dovolene ozdoby, realizovatelnost RGG}

\subsection{Metody vypoctu well-founded modelu: maximal unfounded set, alternujuci pevny bod}

\subsection{Hierarchia semantik, priklady inych semantik}

\subsection{Rozsirenie relacnej algebry o semijoin a antijoin}

\subsection{Vstupne mnoziny, pseudokluce, bezpecnost programov so zabudovanymi (napr. aritmetickymi) predikatmi}

\subsection{Vypocet programov s negaciou zdola nahor (Datalog) a zhora nadol (Prolog, SLD rezolucia)}

\section{Magicke transformacie}

\subsection{Vzory toku dat}

\subsection{Jednoducha magicka transformacia}

\subsection{Zovseobecnena magicka transformacia}

\subsection{Vypocet zovseobecnenych magickych programov}

\subsection{Dosledna rektifikacia: rozbalovanie cyklov}

\subsection{Problem s negaciou}

\section{Optimalizacie na urovni relacnej algebry}

\subsection{Reprezentacie algebraickych vyrazov: stromy, dagy, hypergrafy}

\subsection{Reprezentacia hypergrafu}

\subsection{GYO redukcia hypergrafu}

\subsection{Trhanie usi hypergrafu}

\subsection{Zakony relacnej algebry}

\subsection{Pravidla pre konstrukciu planu vypoctu v relacnej algebre}

\subsection{Optimalizacia poradia joinov}

\subsection{Eliminacia spolocnych podvyrazov}

\subsection{Semantika SQL}

\subsection{Minimalizacia operatorov relacnej algebry}

\subsection{Predpoklady na vypoctovy model, uskutocnitelnost operacii}

\section{Optimalizacie zalozene na redukcii hypergrafu}

\subsection{Reprezentacia dotazu hypergrafom}

\subsection{Transformacia dotazu na standardny hypergraf}

\subsection{Wong-Youssefiho algoritmus, preferencie na poradie ostranovanych hran}

\subsection{Uplny reduktor}
TODO: Bezstratove spojenia: 
$$
\forall i (R_i = \Pi_{R+i}(R_1 \join \ldots \join R_n))
$$
T.j. univerzalna relacia, globalna konzistencia. 

\subsection{Yannakakisov algoritmus}

\subsection{Hypergrafy a nekonecne relacie}

\section{Konjunktivne dotazy}

\subsection{Subsumpcia (pohltenie)}

\subsection{Pohlcujuce zobrazenia}

\subsection{Petrifikovane dotazy}

\subsection{Petrifikovane dotazy}

\subsection{Kanonicke databazy}

\subsection{Hladanie pohlteni, Sariayov algoritmus}

\subsection{Test pohltenia zjednoteni konjunktivnych dotazov}

\subsection{Pohltenie konjunktivneho dotazu programom a opacne}

\subsection{Pohltenie dotazov s negaciou, Levy-Sagivov test}

\subsection{Pohltenie dotazov so zabudovanymi predikatmi, Gupta-Zhang-Ozsoyoglu test}

\section{Minimalizacia konjunktivnych dotazov}

\subsection{Predpoklad univerzalnej relacie}

\subsection{Slabe pohltenie, slaba ekvivalencia konjunktivnych dotazov}

\subsection{Tableaux, konstrukcia tableaux}

\subsection{Minimalizacia tableaux}

\subsection{Uzatvaracia procedura (chase): vynutenie funkcnych zavislosti, multizavislosti, inkluznych zavislosti a joinovacich zavislosti}

\subsection{Silna ekvivalencia konjunktivnych dotazov}

\section{Vyssie normalne formy}

\subsection{Multizavislosti, 4NF}

\subsection{Joinovacie zavislosti, 5NF}

\subsection{Inkluzne zavislosti}

\subsection{Poloformalna metoda "spravneho" navrhu databazy}


 

\end{document}
