\documentclass[10pt,a4paper]{article}
\usepackage[slovak]{babel}
\usepackage[utf8]{inputenc}
\usepackage{amsmath}
\usepackage{amsfonts}
\usepackage{amssymb}
\usepackage[unicode]{hyperref}
\usepackage{graphicx}

\textwidth 6.5in
\oddsidemargin 0.0in
\evensidemargin 0.0in

\title{Poznámky z Úvodu do databázových systémov - materiál na štátnice}
\date{18.06.2012}
\author{Peter Csiba, petherz@gmail.com, \url{https://github.com/Petrzlen/fmfi-poznamky}} 

\begin{document}
\maketitle
\tableofcontents

\clearpage

%%%%%%%%%%%%%%%%%%%%%%%%%%%%%%%%%%%%%%%%%%%%%%%%%%%%%%%%%%%%%%%%%%%%%%%%%%%%%%%%
%%%%%%%%%%%%%%%%%%%%%%%%%%%%%%%%%%%%%%%%%%%%%%%%%%%%%%%%%%%%%%%%%%%%%%%%%%%%%%%%
\section*{Úvod}   

Tento text vysvetľuje netriviálne a nejasné pasáže z predmetu Úvod do diskrétnej matematiky,
prípadne patchuje chyby v skriptách. 

Autor si v matematike a hlavne jej diskrétnej a kombinatorickej časti naozaj verí. 

Nakoniec poznamenajme, že autor sa snažil písať pravdu a len pravdu, keďže jeho odpoveď na záverečných skúškach vychádza z tototo materiálu.
Ak čitateľ chce prispieť ku kvalite textu, nech autorovi napíše a ten mu udelí prístup do repozitára.

%%%%%%%%%%%%%%%%%%%%%%%%%%%%%%%%%%%%%%%%%%%%%%%%%%%%%%%%%%%%%%%%%%%%%%%%%%%%%%%%
%%%%%%%%%%%%%%%%%%%%%%%%%%%%%%%%%%%%%%%%%%%%%%%%%%%%%%%%%%%%%%%%%%%%%%%%%%%%%%%%
\section*{Úvod}   

%%%%%%%%%%%%%%%%%%%%%%%%%%%%%%%%%%%%%%%%%%%%%%%%%%%%%%%%%%%%%%%%%%%%%%%%%%%%%%%%
%%%%%%%%%%%%%%%%%%%%%%%%%%%%%%%%%%%%%%%%%%%%%%%%%%%%%%%%%%%%%%%%%%%%%%%%%%%%%%%%
\clearpage
\section*{Referencie a odporúčaná literatúra}
Obe skriptá sú v repozitáry. 

\begin{itemize}                                
\item Úvod do diskrétnych matematických štruktúr. Daniel Olejár a Martin Škoviera.
\item Úvod do diskrétnych štruktúr. Eduard Toman, BRATISLAVA 2008.
\end{itemize}

\end{document}
