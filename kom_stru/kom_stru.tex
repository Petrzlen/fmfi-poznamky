\documentclass[10pt,a4paper]{article}
\usepackage[slovak]{babel}
\usepackage[utf8]{inputenc}
\usepackage{amsmath}
\usepackage{amsfonts}
\usepackage{amssymb}
\usepackage[unicode]{hyperref}
\usepackage{graphicx}

\textwidth 6.5in
\oddsidemargin 0.0in
\evensidemargin 0.0in

\title{Poznámky z Kombinatorických štruktúr}
\date{17.01.2013}
\author{Peter Csiba, petherz@gmail.com, \url{https://github.com/Petrzlen/fmfi-poznamky}}

\begin{document}
\maketitle
\tableofcontents

\clearpage

%%%%%%%%%%%%%%%%%%%%%%%%%%%%%%%%%%%%%%%%%%%%%%%%%%%%%%%%%%%%%%%%%%%%%%%%%%%%%%%%
\section{Úvod}

Autor neabsolvoval prednášky ani skúšku z predmetu Kombinatorické štruktúry.
Poznámky sú voľným prepisom poznámok Martina Šrámeka doplnených o komentár autora. 

Nakoniec poznamenajme, že autor sa snažil písať pravdu a len pravdu, keďže jeho odpoveď na skúškach vychádza z tototo materiálu.
Ak čitateľ chce prispieť ku kvalite textu, nech autorovi napíše a ten mu udelí prístup do repozitára. 

%%%%%%%%%%%%%%%%%%%%%%%%%%%%%%%%%%%%%%%%%%%%%%%%%%%%%%%%%%%%%%%%%%%%%%%%%%%%%%%%
\section{Latinské štvorce}
\begin{itemize}
\item $n \times n$
\item $\{1,\ldots,n\} = X$
\item Každý riadok aj stĺpec je permutácia. 
\end{itemize}


\begin{itemize}
\item $S_n$ sym. grupa $n!$
\item $\Phi, \Psi$ - permutacie na $X$
\item $\Phi, \Psi$ su \emph{dis...ntne} na $Y$ ak $\forall x \in Y \Psi(x) \neq \Phi(x)$
\item $\Phi, \Psi$ su \emph{dis...ntne} $\Leftrightarrow$ su dis...ntne na $X$
\end{itemize}

\paragraph{Poznamky.}
\begin{itemize}
\item Latinsky stvorec - maximalny latinsky obdlznik.
\item Maximalna mnoznina navzajom maximalne vzdialencyh permutacii. 
\end{itemize}

\subsection{Metrika medzi permutaciami}
\subsubsection{Vseobecna metrika}
\begin{itemize}
\item $S(x,y)=0 \Leftrightarrow x=y$
\item $S(x,y)=S(y,x)$
\item $S(x,y)+S(y,z) \geq S(x,z)$
\end{itemize}
\footnote{
Podla pravidla o generalizacii kde nedavame kvantifikatory, tak su vseobecne.
}

\subsubsection{Metricky system permutacii}
\begin{itemize}
\item $S(\Psi, \Phi)$ - max pocet prvkov mnoziny $Y \subseteq X$ pri ktorej $\Psi$ a $\Phi$ su dis...ntne.
\item $S(\Psi, \Phi) = |\{x \in X, \Psi(x) \neq \Phi(x)\}|$
\item $S(\Psi, \Phi) = S(\Phi^{-1}\Psi, id)$
\end{itemize}

\subsection{Hallova veta}
Latinsky stvorec je maximalna mnozina dis...permutacii z $S_n$. $L_n = [\Phi_1, \ldots, \Phi_n]$.
\paragraph{Hallova veta.}
Nech $(X_1, \ldots, X_k)$ je system mnozin $X_i \subseteq X$. $T \subseteq X$ je system rozlicnych reprezentantov ak $T=[x_1, \ldots, x_k], x_i \in X_i, x_i \neq x_j$ pre $i \neq j$. Potom system $X$ ma system rozlicnych reprezentantov $\Leftrightarrow$ pre kazdy system mnozin $Y \subseteq X$ plati $|\cup Y_i| \geq |Y|$.

\paragraph{Veta.}
Kazdy latinsky obdlznik s $k$ riadkami sa da doplnit na stvorec. Lebo Hallova veta. Presnejsie:
Urobme si bipartitny graf, kde jednu particiu predstavuju stlpce a druhu cisla.
Hrana je medzi cislami, ktore mozeme dat do daneho stlpca.
Tento graf je $n-k$-regularny(zo stlpcovej particie to je jasne a cislo sa mohlo vyskytnut v max
$k$ stlpcoch, takze ma este $n-k$ volnych) a teda ma 1-faktor z ktoreho vieme doplnit dalsi riadok.

\subsection{Normalizovane LS}
\begin{tabular}{l c r}
1 & 2 & \ldots \\
2 & & \\
3 & & \\
\ldots & & \\
\end{tabular}

\subsection{Ortogonalne LS}
\begin{itemize}
\item $L_n = [\Phi_1, \ldots \Phi_n]$ jeden LS
\item $L_n' = [\Phi_1, \ldots \Phi_n]$ sruhy LS
\item $L_n \perp L_n' \Leftrightarrow (i,j) \neq (k,l) \in X \times X$ plati $(\Phi_i(j), \Phi_i'(j)) \neq (\Phi_k(l), \Phi_k'(l))$
\item Tj. vsetky dvojice $(i,j)$. 
\end{itemize}

\paragraph{Vlastnosti 1-2.}
\begin{itemize}
\item $L_n \perp L_n' \Leftrightarrow L_n \cdot L_n'$ je LS.
\item Ak $L_n \perp L_n' \Rightarrow \forall \Phi, \Psi \in S_n: \Psi L_n \perp \Phi L_n' \wedge L_n \Psi \perp L_n' \Phi$.
\end{itemize}

\subsection{Polonormalizovane LS}
$[id, \Phi_2, \ldots, \Phi_n]$. 
\paragraph{Vlastnost 3.}
Nech $L_n^{(1)}, \ldots ,L_n^{(r)}$ je mnozina navzajom $\perp$ LS. Potom $r \leq n-1$. 
TODO - polonormalizovane prelozene cez seba. 

\subsection{Uplna mnozina.}
\paragraph{Uplna mnozina.}
Uplna mnozina $L_n^{(1)}, \ldots ,L_n^{(n-1)}$ je mnozina $n-1$ LS.

\paragraph{Sievers.}
Nech $n=p^r$, kde $p$ je prvocislo a $n \geq 1$. Potom existuje uplna mnozina $(n-1)$ navzajom ortogonalnych LS radu $n$.
TODO - $\exists GF(n) = F$, technicky sporom. 

Basic idea: Zobereme konecne pole $GF(n)$ a polozime $L_a(i,j) = a*i+j$. Zbytok je technicka
dokazovacia otrava.

\section{Vyvazene blokove plany}
\paragraph{$(v,k,\lambda)$-konfiguracia.}
\begin{itemize}
\item $X = \{x_1, \ldots, x_v\}$ - body.
\item System podmnozin $B = {X_1, \ldots, X_n}$ - bloky. 
\item 1. $|X_i|=k$ (konst)
\item 2. $X_i \cap X_j = \lambda$ (konst) $i \neq j$.
\item 3. $0 < \lambda < k < v-1$
\end{itemize}
\paragraph{Incidencna matica.}
$A=(a_{ij}), a_{ij} = 1 \Leftrightarrow x_j \in X_i$
\paragraph{Jednotkova matica.}
$J=(1)$
\subsection{Vlastnosti}
\begin{enumerate}
\item $AJ = kJ$
\item $AA^T = \lambda J + (k-\lambda)I$
\item $det(AA^T)=(det A)^2=[k + \lambda(v-1)](k-\lambda)^{v-1} > 0$, TODO, rozvoj podla riadka
\item $k(k-1) = \lambda(v-1)$, TODO, z $AA^T$ na $JAJ$. 
\item $JA=AJ=kJ$, TODO
\item $AA^T=A^TA$, zamenitelnost blokov a bodov, TODO
\end{enumerate}

\subsection{Bruck, Ryser}
Nutne podmienky na existenciu $(v,k,\lambda)$ konfiguracie:

\begin{itemize}
\item $v$ je parne, tak $k-\lambda$ je stvorec
\item $v$ je neparne, $z^2 = (k-\lambda)x^2 + (-1)^{\frac{v-1}{2}}\lambda yz$ ma nenulove riesenie v $\mathbb{Z}$. 
\end{itemize}

\subsection{Diferencne mnoziny}
Specialny pripad. 

\subsubsection{Definovane na $\mathbb{Z}_v$}
$\mathbb{Z}_v \supseteq D = \{d_1, \ldots, d_k\}$, ak kazdy prvok $a \in \mathbb{Z}_v - {0}$ sa da vyjadrit $\lambda$ roznymi sposobmi ako rozdiel dvoch prvkov z $D$. 
\paragraph{Konstrukcia.}
\begin{itemize}
\item $X=\mathbb{Z}_v$
\item $X_i = D + i$
\end{itemize}
Napriklad $X=\mathbb{Z}_7, D=\{1,2,4\}$ dava $[1,2,4],[2,3,5],[3,4,6],[4,5,0],[5,6,1],[6,0,2],[0,1,3]$ - Fannova rovina $(7,3,1)$. 

\subsubsection{Definovane na grupach}
Nech $G$ je konecna grupa radu $v$, nie nutne komutativna. Mnozina $D = \{d_1, \ldots, d_k\} \subseteq G$ sa nazyva DM zalozena na $G$, ak je splnena jedna z dvoch podmienok: 

\begin{itemize}
\item $\forall a \neq e \, \exists_\lambda (d_i, d_j), i\neq j : a = d_id_j^{-1}$
\item $\forall a \neq e \, \exists_\lambda (d_i, d_j), i\neq j : a = d_i^{-1}d_j$
\end{itemize}

\paragraph{Tvrdenie.} 
Kazda $(v,k,\lambda)$ dif. mnozina zalozena na $G$ definuje $(v,k,\lambda)$ konfiguraciu:
\begin{itemize}
\item $X = G$
\item $B=\{Dg, g \in G\}$
\end{itemize}
TODO

\paragraph{Priklad.}
$(16,6,2)$-konfiguracia. 
\begin{itemize}
\item $G=\mathbb{Z}_4$
\item $D=\{(1,0,0,0),(0,1,0,0),(0,0,1,0),(0,0,0,1),(1,1,0,0),(0,0,1,1)\}$
\end{itemize}

\section{Hadamardove matice}
Len strucne. Nechcelo sa mi texovat matice. 
$H_n = (a_{ij})$
\begin{itemize}
\item (i) $a_{ij} = \pm 1$
\item (ii) $HH^T = nI$ - navzajom ortogonalne, maximalny objem spomedzi jednotkovych
\end{itemize}

\subsection{Vlastnosti}
\paragraph{Zakladne.}
\begin{enumerate}
\item $<H_i, H_j> = n$ ak $i=j$, $0$ ak $i\neq j$. Len iny zapis (ii)
\item $H$ je uzavreta na vymeny riadkov a stlpcov, nasobenie $-1$
\item Kazda $H$ je normalna, tj. $HH^T = H^TH$.
\item Kazdu $H$ maticu je mozne previest na normalnu - prvy riadok a prvy stlpec ma same 1. 
\end{enumerate}

\paragraph{Delitelnost 4.}
$n=1 or n=2 or 4|n$. Normalizacia. Spocitame kolko je typov stlpcov podla prvych torch riadkov. 

\subsection{Konstrukcie}
\subsubsection{Sylvestrova}
\begin{tabular}{c c}
$H$ & $H$ \\
$H$ & $-H$ \\
\end{tabular}
\subsubsection{Kroneckerov sucin}
\begin{tabular}{c c c}
$a_{11}H$ & \ldots & $a_{1n}H$ \\
\ldots & \ldots & \ldots \\
$a_{n1}H$ & \ldots & $a_{nn}H$ 
\end{tabular}

\subsection{Hadamarova hypoteza}
Pre kazde $n$ delitene 4 existuje matica.
Nezname su pre 168,224,284,312. (Najdi bug).  

\subsection{Ekvivalencia s blokovymi planmi}
Normalizovane $H$ matice su ekvivalentne s $(4n-1, 2n-1, n-1)$-konfiguraciami. 
TODO: Trivialne. 

\paragraph{Kvadraticke rezidua.}
TODO. Kvadraticke rezidua -> Diferencna mnozina -> Hadamardova matica. 

\section{Konecne projektivne roviny}
Uz len to najdolezitejsie. 
\begin{itemize}
\item $V_{n+1}(F) = F^{n+1} - {0}$
\item (PP1) Kazdymi dvoma bodmi vedie prave jedna priamka.
\item (PP2) Kazde dve priamky maju prave jeden spolocny bod. 
\item (PP3) Existuju styri rozne body vo vseobecnej polohe (ziadne tri z nich nie su kolinearne)
\end{itemize}

\paragraph{Priklady.}
\begin{itemize}
\item Polgula kde stotoznime priamky cez s bodmi na obale. 
\item $\mathbb{Z}_2^3 - {0}$
\item Znizenim dimenzie. Stotoznime body $y = kx, k \in F$.
\end{itemize}

\subsection{Desarguesova veta}
Ak sú trojuholníky $T1, T2$ perspektivne z bodu $S$, tak su perspektivne aj z priamky. 

\subsection{Vlastnosti}
Nech $n\geq 2, \Pi $ je projektivna geometria. NPSE:
\begin{enumerate}
\item Nejaka priamka obsahuje prave $n+1$ bodov.
\item Nejakym bodom prechadza prave $n+1$ priamok. 
\item Kazda priamka obsahuje presne $n+1$ bodov. 
\item Kazdym bodom prechadza $n+1$ priamok.
\item V $\Pi$ sa nachadza presne $n^2+n+1$ bodov. 
\item V $\Pi$ sa nachadza presne $n^2+n+1$ priamok. 
\end{enumerate}

\paragraph{Oznacenie.}
$n$-rad projektivnej roviny

\subsection{$(v,k,\lambda)$-konfiguracie}
Kazda projektivna rovina, ktora ma na nejakej priamke konecny pocet bodov definuje $(v, k, \lambda)$-konfiguraciu:
\begin{itemize}
\item $v=n^2+n+1$ - body
\item $k=n+1$ - priamka obsahujuca body
\item $\lambda=1$ - prisecniky priamok
\end{itemize}

\subsection{Existencia projektivnej roviny}
\paragraph{Tvrdenie.}
Pre existenciu proj. roviny radu $n$ je nutne, aby pre $n \equiv 1,2 (\textrm(mod) 4)$ existovali $a,b$, take, ze $n = a^2+b^2$. Bez dokazu. 

\paragraph{Hypoteza.}
PR radu $n$ existuje iba pre $n=p^r$. 

\subsection{Ortonormalne latinske stvorce}
\paragraph{Latinska vlastnost.}
Matica $C=(c_{ij})$ rozmerov $n \times (t+2)$ ma \emph{latinsku vlastnost}, ak $(c_{ik}, c_{il}) \neq (c_{jk}, c_{jl})$. 
\paragraph{Lema.}
Nech $n \geq 3$ a $t \geq 2$ su z $\mathbb{N}$. Potom mnozina $t$ navzajom roznych ortogonalnych latinskych stvorcov radu $n$ existuje $\Leftrightarrow$ existuje matica $C=(c_{ij})$, $n\times (t+2)$ s latinskou vlastnostou. 
TODO. Ake su rozmery matice v dokaze? 

\paragraph{Veta.}
Ak existuje mnozina $t$ navzajom ortogonalnych LS radu $n$ a mnozina $t$ ortogonalnych LS radu $m$, tak existuje aj mnozina $t$ OLS radu $nm$.

\paragraph{Dosledok.} $n = p^{\alpha_1} \ldots p^{\alpha_k}$.

\subsubsection{LS a PR}
\paragraph{Veta.}
Nech $n \geq 3$. Potom PR radu $n$ existuje $\Leftrightarrow$ existuje $n-1$ navzajom ortogonalnych LS radu $n$. 

$\Rightarrow$. Fixujeme jednu priamku $X={x_1, \ldots, x_{n+1}}$. Zvysnych $n^2$ bodov oznacime $y_1,\ldots,y_{n^2}$. Priamky prechadzajuce $x_j$ oznacime postupne $L_{j1},\ldots,L_{jn}$. Potom $c_{ij}=k \Leftrightarrow y_i \in L_{jk}$. Sporom. 
$\Leftarrow$. Majme $n-1$ OLS radu $n$ a skonstruujeme $C$ rozmerov $n^2 \times (n+1)$ s LV. Bod a hodnota v $C$ urcuju na ktorej priamke lezi $y_i \in L_{jk}$. 
TODO. 

\subsection{Singerove diferencne mnoziny}
PG. Kvadraticke rezidua. Bikvadraticke rezidua. Tetrakvadraticke rezidua.


\section{Nevyvazene blokove plany} 
Nevyvazena $(b,r,v,k,\lambda)$-konfiguracia je system podmnozin-blokov $\{X_1, \ldots, X_b\}, X_i \subseteq X$, kde $X = \{x_1,\ldots, x_v\}$ a plati:
\begin{enumerate}
\item $|X_i|=k$
\item $x_i$ sa vyskytuje prave v $r$ blokoch 
\item $x_i, x_j$ sa spolocne vyskytuju v $\lambda$ blokoch 
\item $0 < \lambda, k < v-1$ (netrivialnost)
\end{enumerate}

\paragraph{Steinerovske systemy trojic.}
$k=3, \lambda=1$. 
\paragraph{Graf $K_v^\lambda$.}
Kompletny graf o $v$ vrcholoch s $\lambda$ nasobnymi hranami. $(b,r,v,k,\lambda)$-konfiguracia odpovedaju jej rozkladu. 

\subsection{Vlastnosti}
Incidencna matica $b \times v$.
\begin{enumerate}
\item $AJ_v = kJ_{b,v}$
\item $J_bA = rJ_{b,v}$
\item $AA^T = \lambda J_v - (r-\lambda)I_v$
\item $bk = vr$. Zratame dvojice dvoma roznymi sposobmi. 
\item $r(k-1)=\lambda (v-1)$. Zratame dvojice s fixnym prvkom. 
\item $det(A^TA = (r + \lambda (v-1))(r-\lambda)^{v-1}$
\item $b \geq v$ (Fischerova nerovnost). Dokaz z predoslych dvoch. Dosledok $r \geq k$.  
\end{enumerate}

%%%%%%%%%%%%%%%%%%%%%%%%%%%%%%%%%%%%%%%%%%%%%%%%%%%%%%%%%%%%%%%%%%%%%%%%%%%%%%%%%%%%%%
\section{Steinerovske systemy trojic}
SST je dvojica $S = (P,B)$ kde 
\begin{itemize}
\item $|P|=v$
\item $B$ ke system trojprvkovych podmnozin $P$ takych, ze $\forall \{x_i, x_j\} \in {P \over 2}$ patri prave do jednej trojice. 
\end{itemize}
Zjavne blokovy plan je $k=3$ a $\lambda=1$. Mozeme nahliadnut na SST ako na rozklad $K_v$ na $K_3$. 

\subsection{Veta Kirkman}
\paragraph{Tvrdenie.} 
Nutne $v \equiv 1,3 (\textrm{mod} 6)$
\paragraph{Veta.} 
Pre kazde $v \equiv 1,3 (\textrm{mod} 6)$ existuje SST. $N(v) \geq (e^{-5}v)\frac{v^2}{6}$.
Bez dokazu.

\subsection{Konstrukcie}
\subsubsection{Projektivne SST}
\begin{itemize}
\item $\mathbb{Z}_2^{n+1} - {0}$ -> body $PG(2,n)$
\item bloky $\{x,y,z\}$, $x+y+z = 0$ -> priamky v PG(2,n)
\end{itemize}
TODO

\subsubsection{Afinne SST}
\begin{itemize}
\item $\mathbb{Z}_3^{n}$ - body $AG(3,n)$
\item bloky $\{x,y,z\}$ - priamky v PG(2,n)
\end{itemize}
TODO


\subsubsection{Priamy sucin SST}
\begin{itemize}
\item $R=(P,B)$
\item $S=(Q,C)$
\item $R \times S = (P \times Q, D)$
\end{itemize}

D obsahuje bloky v jednom z troch tvarov.
TODO

\subsubsection{$2n+1$ konstrukcia}
TODO

\subsubsection{Wilsonova-Schreiberova konstrukcia}
\begin{itemize}
\item Abelovska grupa $A$ radu $n$.
\item $P=A \cup \{\alpha, \beta\}$
\item TODO
\item TODO
\end{itemize}

\subsection{Ciastocny SST}
Pozadujeme, aby kazda dvojica bodov bola nanajvys v jednej trojici. 
\paragraph{Tvrdenie.} Kazdy SST sa da (s pridanim nejakeho poctu bodov) doplnit na SST. TODO: niekde chyba ciastocny. 

\subsection{T-design}
Steinerovsky system $S(t,k,n)$ je $t$-blokovy plan, taky, ze $\lambda=1$. (System $k$-prvkovych podmnozin $n$-prvkovej mnoziny taky, ze kazda $t$-prvkova podmnozina je obsiahnuta v prave jednom bloku.) Navyse musi platit $1 < t < k < n$.
\subsubsection{Steinerovske systemy stvoric}
$S(3,4,n)$
\subsubsection{Steinerovske systemy petic}
$S(4,5,n)$

\subsection{Projektivne specialne linearne grupy}
$k$-tranzitivita, ostra $k$-tranzitivita. TODO.

\paragraph{Tvrdenie (Klasifikacia $K \cup G$).}
TODO

\subsubsection{Mathieuove grupy}
TODO

\section{Symetricke konfiguracie} 
\begin{itemize}
\item Kazdym bodom prechadza $k$ priamok (kazdy bod je v $k$ blokoch)
\item Kazda priamka prechadza $k$ bodmi (velkost bloku je $k$)
\end{itemize}

\subsection{Napriklad}
\begin{itemize}
\item $7_3$ - Fannova rovina
\item $8_3$ - Mobius-Kantor
\item $9_3$ - Pappus z Alexandrie
\item $10_3$ - Desargues
\item $15_3$ - Cremona-Richmond
\end{itemize}
TODO 

\section{Matroidy}
Axiomatizacia linearnej nezavislosti. 

\subsection{Definicia}
\begin{itemize}
\item $X$ - konecnorozmerny vektorovy priestor. 
\item $A \subseteq X$ - nezavisla mnozina vektorov. 
\item (i) $A < \textrm{alef}_0$
\item (ii) $A$ je LN $\Rightarrow A' \subseteq A$ je LN 
\item (iii) $\emptyset$ je LN (trivialny dosledok (ii))
\item (iv) $|A_1| < |A_2| \Rightarrow \exists x\in A_2 - A_1 : A_1 \cup \{x\}$ je LN
\end{itemize}

\paragraph{Matroid.}
Nech $X$ je konecna mnozina, $N \subseteq P(X)$. Potom $(X,N)$ je matroid, ak plati: 
\begin{itemize}
\item N0) $\emptyset \in N$
\item N1) $\forall A \in N: A' \subseteq A \Rightarrow A' \subseteq N$ (dedicnost)
\item N2) $\forall A,B \in N: |A|<|B| \Rightarrow \exists x \in B-A : A \cup {x} \in N$. 
\end{itemize}

\subsection{Specialne matroidy} 
\paragraph{Linearny matroid.} 
\begin{itemize}
\item $V = \{ R_1, \ldots, R_n\}$ vektory nad polom $F$. 
\item Nech $X = \{1, \ldots, n\}$
\item $\forall A: A \subseteq X \Rightarrow (A \in N \Leftrightarrow \{R_i, i \in A\}$ je LN)
\item Potom $(Xn, N)$ je \emph{linearny matroid}. 
\end{itemize}

\paragraph{Grafovy matroid.}
\begin{itemize}
\item $X = E(G)$
\item $A$ je acyklicka
\end{itemize}

\subsection{Vlastnosti} 
\paragraph{Tvrdenie.} Nech $N_1, N_2$ su maximalne matroidy vzhladom na inkluziu. Potom $|N_1|=|N_2|$. 

\paragraph{Baza matroidu.} Bazou matroidu nazyvame kazdu maximalnu nezavislu mnozinu vzhladom na inkluziu. 


\paragraph{Veta.} Nech $(X, S)$ je lubovolny system (asi ze $S \subseteq P(X)$). NPSE: 
\begin{itemize}
\item $(X,S)$ je matroid. 
\item $S$ je neprazdny dedicny system (t.j. NO, N1) splnajuci podmienku N2': $\forall A \subseteq X : \forall $ maximalne $B \subseteq A, B \in S$ maju rovnaku mohutnost. Dokaz: ked nie, tak vieme doplnit. 
\end{itemize}

\subsection{Hodnotova funkcia} 
$r_u : P(X) \rightarrow \mathbb{N}, r_u(A) = $ najvecsia mohutnost nezavislej mnoziny v $A$. 

\paragraph{Veta.} Nech $M=(X,N)$ je matroid, $r$ je hodnotova funkcia. Potom plati: 
\begin{itemize}
\item 
\item
\item
\end{itemize}

\begin{itemize}
\item
\item
\end{itemize}

\begin{itemize}
\item
\item
\end{itemize}

\end{document}
