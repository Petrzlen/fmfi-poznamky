\documentclass[12pt,a4paper]{article}

\usepackage[slovak]{babel}
\usepackage[utf8]{inputenc}
\usepackage[T1]{fontenc}

\usepackage{listings}
\usepackage{graphicx}
\usepackage{tabularx} 

\usepackage{hyperref} 

\usepackage{amsmath} 

\lstset{
language=sh
,breaklines=true
,basicstyle=\ttfamily
, showstringspaces=false}

\author{Peter Csiba}
\textwidth 6.5in
\oddsidemargin 0.0in
\evensidemargin 0.0in

\title{Pravdepodobnostné metódy - Domáca úloha \#3 - Minimálny rez}
\date{04-04-2014}
\author{Peter Csiba, csiba4@uniba.sk}

\begin{document}
\maketitle

\section*{Zadanie}
Uvažujme nasledujúci algoritmus pre hľadanie minimálneho rezu: Náhodne ováhujeme hrany a spočítame minimálnu kostru $T$. Následne odstránime najťažšiu hranu v kostre - tým sa $T$ rozpadne na dva stromy - $T_1$ a $T_2$. Dokážte, že s pravdepodobnosťou $\Omega(1/n^2)$ budú množiny vrcholov $T_1$ a $T_2$ zodpovedať množinám vrcholov minimálneho rezu. (To znamená, ak postup zopakujeme $O(n^2log n)$-krát, s vysokou pravdepodobnosťou nájdeme minimálny rez.) [3b]

\paragraph{Riešenie.}
Na druhej strane. 

\paragraph{Zdroje.} 
\begin{itemize} 
\item \url{http://en.wikipedia.org/wiki/Kruskal's_algorithm} 
\item \url{https://docs.google.com/file/d/0BzME2WFG2071R3JSQjhIb1JtVnM/edit} 
\end{itemize} 

\newpage
\section*{Riešenie}
Uvažujme Kruskalov algoritmus na hľadanie minimálnej kostry: 

\begin{lstlisting} 
 1  A = emptyset
 2   foreach  v in G.V:
 3     MAKE-SET(v)
 4   foreach  (u, v) ordered by weight(u, v), increasing:
 5     if  FIND-SET(u) neq FIND-SET(v):
 6       A = A union {(u, v)}
 7       UNION(u, v)
 8  return  A
\end{lstlisting}

Naše riešenie bude postupovať podľa algoritmu Karger-a a Stein-a odprednášaného na prednáške: 
\begin{enumerate} 
\item kým G má viac ako dva vrcholy: 
  \begin{enumerate} 
  \item vyber náhodnú hranu $e \in_R E(G)$; 
  \item kontrahuj $G \Leftarrow G/e$; 
  \item algoritmus dá na výstupe rez $C$ práve vtedy, keď neskontrahujeme žiadnu hranu z $E(C, V/C)$; 
  \end{enumerate} 

\item nech $C^*$ je minimálny rez s hodnotou $c^*$; žiadny vrchol nemá stupeň nižší ako $c^*$; hrán je aspoň $nc^*/2$.
\item v $i$-tej iterácií máme $n_i = n - i + 1$ vrcholov; 
ak sme zatiaľ neskontrahovali žiadnu hranu z $C^*$ 
tak minimálny rez je stále $C^*$ a počet hrán je aspoň $nc^*/2$. 
\item $Pr[$v $i$-tom kroku skontrahujeme hranu z $C^* \mid $ prvých $i - 1$ krokov sme žiadnu hranu z $C^*$ neskontrahovali $] \leq 2 / n_i$.  
\item $Pr[$neskontrahujeme žiadnu hranu z $C^*] = Pr[$ výstup algoritmu je $C^*] \geq \Pi_{i=1}^{n-2} (1 - 2/n_i) = 1 / \binom{n}{2}$. Takže ľubovoľný konkrétny minimálny rez $C^*$ dostaneme na výstupe s pravdepodobnosťou $\Omega(1/n^2)$.
\end{enumerate} 

Ak teraz spojíme myšlienky Kruskalovho algoritmu na minimálnu kostru a pravdepodobnostného algoritmu Karger-a a Stein-a na hľadanie minimálnej kostry tak de facto máme riešenie našej domácej úlohy. 

Vidno to z dvoch krokov: 
\begin{itemize}
\item Kruskal(7)=UNION(u, v) je vlastne kontrakcia (1b), lebo komponenty=SETy prislúchajúce $u,v$ sa spoja a všetky ďalšie uvažované hrany sa vynechajú. 
\item Kruskal(4)=vyber najlacnejšiu hranu mimo kontrahovaných komponentov je (1a)=vyber náhodnú hranu
\end{itemize} 

\end{document}

