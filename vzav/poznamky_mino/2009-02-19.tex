\documentclass{article}
\usepackage{slovak}
\usepackage{amssymb}
\usepackage[utf8]{inputenc}
\usepackage[T1]{fontenc}
\usepackage{listings}
\begin{document}

\paragraph{Veta (o pamäťovej hierarchii):} Nech $\log n \le S_1(n) = o(S_2(n))$,
kde $S_2(n)$ je páskovo konštruovateľná. Potom $DSPACE(S_1(n)) <
DSPACE(S_2(n))$.

\paragraph{Dôkaz:} \emph{Prefixový kód} Turingovho stroja $M$ je reťazec tvaru
$11\cdots1<M>$. Nech $M'$ je deterministický Turingov stroj, ktorý:
\begin{enumerate}
	\item Na vstupe $w$ ($\mid w 
		\mid = n$) označí na pracovných páskach pamäť rozsahu $S_2(n)$.
		Ak by mal $M'$ použiť v ďalšom pamäť $ > S_2(n)$, potom výpočet
		zastaví a neakceptuje vstup.
	\item Ak vstup $w$ nie je prefixový kód žiadneho Turingovho stroja,
		potom $M'$ neakceptuje $w$.
	\item $w$ je prefixový kód nejakého Turingovho stroja $M$. Potom $M'$
		(rešpektujúc 1.) simuluje $2^{S_2(n)}$ krokov stroja $M$ na $w$
		a $M'$ akceptuje $w$ práve vtedy, keď $M$ neakceptuje $w$ počas
		$2^{S_2(n)}$.
\end{enumerate}

\par $M$ -- ľubovoľný deterministický Turingov stroj s páskovou zložitosťou
$S_1(n)$.

\paragraph{Dokážeme:} $M$ neakceptuje $L(M) \in DSPACE(S_2(n))$, teda
$L(M') \notin DSPACE(S_1(n))$ $\Rightarrow$ $DSPACE(S_1(n)) \subsetneq
DSPACE(S_2(n))$.
\par Nech má $M$ $k$ páskových symbolov a $q$ stavov. Pre dostatočne dlhý vstup
$w$ ( $\mid w\mid = n$) platí: \\
	$M$ akceptuje $w$ práve $\iff$ $M$ akceptuje $w$ počas $\leq
	\left(q(n+2)t^{S_1(n)}(S_1(n)+1\right)^k$ $\leq 2^{S_2(n)}$ krokov (lebo $\log n
	\leq S_1(n) = o(S_2(n))$)

	\par Pre dostatočne dlhý prefixový kód stroja $M$ platí vlastnosť
	\ref{itm:1a} pre $m =
	\mid w \mid$:
	\begin{enumerate}
		\item \label{itm:1a} $M'$ je schopný -- neprekročiac $S_2(n)$ --
			simulovať $2^{S_2(n)}$ krokov výpočtu stroja $M$ na $w$ (lebo
			$S_1(n) = o(S_2(n))$).
			$(iii)$ a \ref{itm:1a} $M'$ akceptuje $w$ $\iff$ $M$ neakceptuje
			$w$ počas $2^S_2(n)$ $\iff$ $M$ neakceptuje $w$, t.j. $M$
			neakceptuje $L(M')$.
	\end{enumerate}

\paragraph{Definícia:} Funkcia $T(n)$ je \emph{časovo konštruovateľná}, ak
existuje deterministický Turingov stroj, ktorý na každom vstup dĺžky $n$ vykoná
práve $T(n)$ krokov.

\paragraph{Veta (o časovej hierarchii):} Nech  $n \leq T_1(n) = o(T_2(n)/\log
T_1(n))$, kde $T_2(n)$ je časovo konštruovateľná. Potom $DTIME(T_1(n))
\subsetneq DTIME (T_2(n))$.

\paragraph{Veta (Savitch):} Nech $S(n) \geq \log n$ je páskovo konštruovateľná.
Potom $NSPACE(S(n)) \subseteq DSPACE(S^2(n))$.

\paragraph{Dôkaz:} $M$ -- ľubovoľný nedeterministický Turingov stroj s pamäťovou
zložitosťou $S(n)$ majúci $k$ pások, $q$ stavov, $t$ páskových symbolov. Na
vstupe dĺžky $n$ sa $M$ môže dostať do najviac $q(n+2)\left(
(S(n)+1)t^{S(n)}\right)^k \leq c^{S(n)}$ rôznych konfigurácií pre vhodnú
konštantu $c$ (nazývame ich ohraničené konfigurácie). Potom ak $x \in L(M)$,
potom existuje akceptačný výpočet stroja $M$ na $x$ majúci %$\leq c^{S(\mid x \mid)}$ 
krokov (prečo je to tak? -- otázka na skúške).

\par $M$ sa dostane z konfigurácie $c_1$ do konfigurácie $c_2$ počas
$\leq i$ krokov $\iff$ $M$ sa dostane z $c_1$ do nejakej $c_3$ počas najviac $\leq
\lfloor i/2 \rfloor$ a z $c_3$ do $c_2$ počas $\leq \lceil i/2
\rceil$ krokov.

\par Algoritmus na simulovanie stroja $M$:
\begin{verbatim}
begin
  nech x je vstup dĺžky n a nech c_0 je príslušná počiatočná konfigurácia stroja
  $M$ pre každú S(n)-ohraničenú akceptačnú konfiguráciu C_f vykonaj:
  if TEST(c_0, C_f, c^{S(n)} then accept
end

procedure TEST(c_1,c_2,i); {TEST zistí, či M prejde z c_1 do c_2 počas \leq i
krokov}
if c_1 = c_2 alebo $M$ prejde z c_1 do c_2 v jednom kroku then return true
if i>1 then
begin
  pre každú S(n)-ohraničenú c_3 vykonaj
    if TEST(c_1,c_3,\lceil i/2 \rceil) and TEST (c_3,c_2, \lfloor i/2
    \rfloor) then return true
  end
  return false
end
\end{verbatim}

\par Uvedený algoritmus možno implementovať na deterministickom Turingovom
stroji $M'$, ktorý:
\begin{itemize}
	\item jednu z pások používa ako zásobník
	\item pri každom volaní procedúry $TEST$ pridá do zásobníka 1 blok
		obsahujúci: $c_1$, $c_2$, $z$, $c_3$, $TEST(c_1, c_3, \lceil i/2
		\rceil)$, $TEST(c_1, c_3, \lfloor i/2 \rfloor)$ a adresa návratu
		pre volanie procedúry $TEST$ (t.j. pre volanie medzi $if$ a
		$and$ alebo $and$ a $then$.
	\item po vykonaní jedného volania procedúry $TEST$ odstráni príslušný
		blok zo zásobníka.
\end{itemize}
Rozsah bloku je teda $O(S(n))$, lebo $i \leq c^S(n)$.
\par Hĺbka rekurzie (počet blokov v zásobníku je najviac $1+\log_2 c^{S(n)} =
O(S(n))$. lebo tretí parameter procedúry $TEST$ je zmenšený zhruba na polovicu
pri každom volaní.

\par Z predchádzajúcich úvah vyplýva, že $M'$ je stroj s pamäťovou zložitosťou
$O(S^2(n))$. $M'$ možno prerobiť (kompresia pásky) na deterministický Turingov
stroj s páskovou zložitosťou $S^2(n)$ akceptujúci rovnaký jazyk.

\paragraph{Veta:}
\begin{enumerate}
	\item \label{itm:2a} $DTIME(T(n)) \subseteq NTIME(T(n)) \subseteq
		DSPACE(T(n))$ pre všetky časovo konštruovateľné $T(n)$.
	\item $DSPACE(S(n)) \subseteq NSPACE(S(n)) \subseteq \bigcup_{c>0}
		DTIME(c^{\log n + S(n)})$ pre všetky páskovo konštruovateľné
		$S(n)$.
\end{enumerate}

\paragraph{Dôkaz:}

\subparagraph {Pre $NTIME(T(n)) \subseteq DSPACE(T(n))$:}. Vyberieme ľubovoľný nedeterministický
Turingov stroj $M$ s časovou zložitosťou $T(n)$ majúci najviac $d$ možností pre
ďalší krok. $M'$ je deterministický storj s pamäťovou zložitosťou $T(n)$, ktorý
na vstupe $x$ na jednej z pások postupne generuje všetky možné reťazce dĺžky
najviac $T(\mid x \mid)$ nad abecedou $\{ a_1, \cdots, a_d \}$. Tieto reťazce
reprezentujú možné nedeterministické výpočty stroja $M$ na $x$. $M'$ ich
simuluje a $M'$ akceptuje $x$ $\iff$ $M$ akceptoval $x$ v niektorom výpočte. Z
toho vyplýva, že $L(M) = L(M')$. 

\subparagraph{Pre $NSPACE(S(n)) \subseteq \bigcup_c DTIME(c^{\log n + S(n)})$}. 
Nech $M$ je ľubovoľný nedeterministický Turingov stroj s pamäťovou
zložitosťou $S(n)$, majme $q$ stavov, $t$ páskových symbolov, $k$ pások. $x$ --
vstup pre $M$, $\mid x\mid = n$.
$$
\begin{array}{lll}
	G & = &(V,E) \mbox{ -- orientovaný graf} \\
	V & = & \{c \mid c \mbox{ je konfigurácia stroja $M$ max}; \\
	\mid V \mid & \leq & (n+2)( (S(n)+1)t^{S(n)} ) \leq d^{\log n + S(n)} \\
	E & = & \{ (c,c') \mid \mbox{$M$ prejde v 1 kroku z konfigurácie $c$ do
	$c'$ } \}\\
	x & \in & L(M) \iff \mbox{v $G$ existuje cesta z počiatočnej
	konfigurácie do niektorej akceptačnej konfigurácie}
\end{array}
$$

To, či v $G$ taká cesta existuje, možno zistiť deterministickým Turingovým
strojom pomocou vhodne implmementovaného (napr. Dijkstrovho) algoritmu v čase
najviac $\mid V \mid \leq (d^{\log n + S(n)}) = c^{\log n + S(n)}$ pre vhodné
$m$ a $c=d^m$.
\end{document} 
