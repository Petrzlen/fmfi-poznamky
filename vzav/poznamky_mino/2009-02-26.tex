\documentclass{article}
\usepackage{slovak}
\usepackage[utf8]{inputenc}
\usepackage[T1]{fontenc}
\usepackage{amssymb}
\begin{document}
\paragraph{Dôsledôk:} $DSPACE(\log n)\subseteq NSPACE(\log n) \subseteq P
\subseteq NP = PSPACE \subseteq EXP \subseteq NEXT$ --
druhá inklúzia vyplýva z predchádzajúceho tvrdenia (b), tretia z (a), rovnosť zo
Savitchovej vety, predposledná inklúzia opäť z (b).

\paragraph{Poznámka:} Nie je známe, či
\begin{itemize}
	\item $DSPACE(\log n) \subsetneq NP$
($P=^?NP$). 
	\item $P \subsetneq PSPACE$. ($P=^?NP$).
	\item $NP \subsetneq NEXP$. 
\end{itemize}

Je známe, že:
\begin{itemize}
	\item $NSPACE(\log n) \subsetneq PSPACE$ (Savitch a
pamäťová hierarchia). 
	\item $P \subsetneq EXP$ (časová hierarchia).
\end{itemize}

\section{Translačná lema}
\paragraph{Tvrdenie:} Ak $NSPACE(n^4) \subseteq NSPACE(n^3)$, potom
$NSPACE(n^{20}) \subseteq NSPACE(n^{15})$.
\paragraph{Dôkaz:} Nech $L_1 \in NSPACE(n^{20})$, teda existuje
nedeterministický Turingov stroj $M_1$ akceptujúci $L_1$ s pamäťou $n^{20}$.
$$
	L_2 = \{ x\#^i \mid x \in L_1 \land |x|^5\}
$$

Nech $M_2$ je nedeterministický Turingov stroj, ktorý na vstupe  $x\#^i$ najprv
overí, či $|x\#^i| = |x|^5$. Ak áno, potom sa na vstupe $x\#^i$ správa rovnako,
ako $M_1$ na $x$.

\par $M_2$ akceptuje $L_2$ s pamäťou $n$, lebo $|x\#^i|^4 = (|x|^5)^4 = x^{20}$ a
$M_1$ akceptuje $L_1$ s pamäťou $n^{20}$. $L_2 \in NSPACE(n^4)$. Z predpokladu
tvrdenia vyplýva $L_2 \in NSPACE(n^3)$, teda existuje nedeterministický Turingov
stroj $M_3$ akceptujúci $L_2$ s pamäťou $n^3$. 

\par Nech $M_4$ je nedeterministický Turingov stroj, ktorý má na vstupe $x$.
Najprv zostrojí (na niektorej pracovnej páske) reťazec $x\#^i$, $|x\#^i| = |x|^5$.
Potom sa na "vstupe" $x\#^i$ správa rovnako, ako $M_3$ na $x\#^i$. $M_4$ akceptuje
$L_1$ s pamäťou $n^{15}$, lebo $|x|^{15} - (|x|^5)^3 = |x\#^i|^3$ a  $L_1 \in
NSPACE(n^{15})$.

\paragraph{Lema (translačná):} Nech $S_1(n)$, $S_2(n)$, $f(n)$ sú páskovo
konštruovateľné, $S_2(n) \geq n$, $f(n) \geq n$. Potom $NSPACE(S_1(n)) \subseteq
NSPACE(S_2(n)) \Rightarrow NSPACE(S_1(f(n))) \subseteq NSPACE(S_2(f(n)))$.

\paragraph{Tvrdenie:} $NSPACE(n^3) \subsetneq NSPACE(n^4)$.
\paragraph{Dôkaz:} Nech by $NSPACE(n^4) \subseteq NSPACE(n^3)$. Aplikáciou
translačnej lemy pre $f(n)=n^3$, $f(n)=n^4$, $f(n)=n^5$ a $S_1(n)=n^4$,
$S_2(n)=n^3$ postupne dostaneme:
$$ 
\left.
\begin{array}{lll}
	NSPACE(n^{12}) & \subseteq & NSPACE(n^9) \\
	NSPACE(n^{16}) & \subseteq & NSPACE(n^{12}) \\
	NSPACE(n^{20}) & \subseteq & NSPACE(n^{15}) \\
\end{array} 
\right\}
NSPACE(n^{20}) \subseteq NSPACE(n^9).
$$

Zo Savitchovej vety a vety o pamäťovej hierarchii platí
$$
\left.
\begin{array}{lll}
	NSPACE(n^9) &\subseteq& DSPACE(n^{18}) \\
	DSPACE(n^{18}& \subsetneq& DSPACE(n^{20}) \\
\end{array}
\right\} NSPACE(n^{20}) \subsetneq DSPACE(n^{20}) \subseteq NSPACE(n^{20})
$$, čo je spor.

\paragraph{Veta:} Ak $\varepsilon > 0$ a $r \geq 0$, potom $NSPACE(n^r)
\subsetneq NSPACE(n^{r+\varepsilon})$.

\section{Vety o medzere a o zrýchľovaní}
\paragraph{Tvrdenie:} Existuje rekuzrívna funkcia $S(n) \geq n$, teda
$DSPACE(S(n))=DSPACE(2^{S(n)})$.
\paragraph{Dôkaz:}Nech $m \in N$, $M$ je ľubovoľný deterministický Turingov
stroj, $x$ je ľubovoľný vstup, $s$ je maximálna veľkosť použitej pamäte počas
výpočtu $M$ na $x$; $s \in N \cup \{\infty\}$.

$$
TEST(m,M,x) = \left\{ \begin{array}{ll}
1 & \mbox{ak}~s \in <m+1,2^m> \\
0 & \mbox{ak}~s \notin <m+1,2^m> \\
\end{array}
\right.
$$

\subparagraph{Otázka:} Existuje algorigmus počítajúci $TEST \forall m,M,x$ ?
\subparagraph{Odpoveď:} Áno.
\paragraph{Dôvod:} Označme $t$ počet rôznych konfigurácií, do ktorých sa môže
dostať $M$ na $x$ použijúc najviac pamäť $2^m$. Simulujme $M$ na $x$ počas $t+1$
krokov, pričom existujú 3 prípady:
\begin{enumerate}
	\item $M$ sa zastavil na $x$ do času $t$, teda vieme určiť $s$ a teda aj
	$TEST(m,M,x)$. V čase $t+1$ má stroj $M$ použitú pamäť najviac $2^m$.
	\item V čase $t+1$ má $M$ použitú pamäť $\leq 2^m$, potom platí to isté,
	ako v predchádzajúcom prípade.
	\item V čase $t+1$ má $M$ použitú pamäť $> 2^m$, a teda $TEST(m,M,x)=0$,
	lebo $s>2^M$ (aj keď nepoznáme $s$).
\end{enumerate}
Nech $M_1, M_2, ...$ je ľubovoľná efektívne očíslovaná postupnosť všetkých
Turingovych strojov (t.j. z čísla $i$ vieme algoritmicky zostrojiť $M_i$.).
\paragraph{Algoritmus pre $S(n)$}: Nech $s_1 < s_2 < ... < s_p$ sú všetky
veľkosti použitej pamäte v čase zastavenia sa niektorého stroja $M_i$, $i \leq
n$ na niektorom $x \in \{0, 1\}^n$.
\paragraph{Cieľ:} Určiť $S(n)$ tak, aby platilo $S_i \notin <S(n)+1, 2^S(n)>$
pre $i=1, 2, \ldots, p$. Hľadáme dosť veľký interval, do ktorého nepadne žiadne
$s_i$.
\par
Pre $m \leftarrow n, n+1, \ldots$, ak $TEST(m,M,x) = 0 \forall M_j (j \leq n)
\land x \in \{0, 1\}^n$, potom $S(n) = m$. (také $m$ existuje, lebo pre každú
dvojicu $(M_j,x)$ (ktorých je $n2^n$, teda konečne veľa) platí $TEST(m,M,x)=1$
len pre konečne veľa $m$ (teda $m < s \leq 2^m$), ale algoritmus skúša nekonečne
veľa $m\geq n$).

\par Nech $L \in DSPACE(2^{S(n)})$, potom existuje $M_k$ akceptujúci $L$ s
pamäťou najviac $2^S(n)$. Nech $n \geq k$, $x \in \{ 0, 1 \}^n$. $M_k$ je
deterministický, a teda $M_k$ zastaví na $x$ s použitou pamäťou $S_j \leq
2^{S(n))}$, $1 \leq j \leq p$. Potom $S_j \leq S(n)$, teda $M_k$ sa zastaví na
každom $x$ dĺžky $k$ s pamäťou $\leq S(n)$, teda $L \in DSPACE(S(n))$.

\paragraph{Veta (Gap theorem):} Pre každú rekurzívnu funkciu $g(n) \geq n$
existuje rekurzívna funkcia $S(n) \geq n$ taká, že
$DSPACE(S(n))=DSPACE(g(S(n)))$.

\paragraph{Veta (Speed up theorem):} Pre každú rekurzívnu funkciu $f(n) \geq
n^2$ existuje rekurzívny jazyk $L$ taký, že pre ľubovoľný deterministický
Turingov stroj $M$ akceptujúci $L$ v čase $T(n)$ existuje deterministický
Turingov stroj $M'$ akceptujúci $L$ v čase $T'(n)$ tak, že platí
$$
	\mathcal{F}(T'(n)) \leq T(n)
$$
skoro pre všetky $n$.
\paragraph{Poznámka:} Napríklad pre $f(n)=2^n$ platí: $2^{T'(n)} <T(n)$, teda
$T'(n) \leq \log T(n)$.
\end{document}
