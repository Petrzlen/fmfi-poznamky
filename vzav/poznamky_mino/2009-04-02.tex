\documentclass{article}
\usepackage[utf8]{inputenc}
\usepackage{slovak}
\usepackage{amsmath}
\usepackage{amssymb}
\begin{document}

\paragraph{Označenie}
\begin{enumerate}
	\item $rand$ -- vráti náhodný real $y \in <0,1)$
	\item $grand(a,b)$ -- vráti náhodný real $y \in <a,b), a,b \in
	\mathcal{R}$
	\item $irand(i,j)$ -- vráti náhodné cele $y \in <i,j>, i,j\in
	\mathcal{Z}$
	\item $brand$ -- \uv{hodí mincou} a vráti $0$ alebo $1$.
\end{enumerate}

\paragraph{Poznámka} Pomocou hociktorej procedúry (1) -- (4) je možné zostrojiť
ostatné:
\begin{itemize}
	\item $grand(a,b) = a + (b-a)*rand$
	\item $irand(i,j) = \lfloor grand(i,j+1)\rfloor$
	\item $brand=irand(0,1)$
	\item $$brand = \left\{ 
		\begin{array}{ll}
			0 & rand \mbox{vrátí} y < \frac{1}{2} \\
			1 & \mbox{inak} \\
		\end{array} \right.
		$$
\end{itemize}

\begin{verbatim}
procedure rand
vykonaj brand r-krat s vysledkami b_1, b_2, ... b_r (b_i \in {0,1} \forall i)
return y=0.b_1b_2...b_r (real.y\in<0,1) v bin. tvare).
end;
\end{verbatim}

\paragraph{Definícia} \emph{Pravdepodobnostný algoritmus} akceptuje jazyk $L \in
\Sigma^*$ s chybou $\varepsilon (0<\varepsilon<\frac{1}{2})$ ak $\forall x \in
\Sigma^*$ platí: Pravdepodobnosť toho, že $A$ akceptuje $x \notin L$ (zamietne
$x \in L$) je $ \leq \varepsilon$ (t.j. $A$ akceptuje $x \in L$ (resp. zamietne
$x \notin L$) s pravdepodobnosťou $\geq 1 - \varepsilon$).

\paragraph{Príklad} Nech $A$ používa $brand$.

%pic1

Strom pravdepodobnostného výpočtu $A$ $x$ vrcholy -- volania $brand$; $a$ --
akceptuj, $z$ -- zamietni.

Do konkrétneho listu s hĺbkou $l$ sa$A$ dostane s pravdepodobnosťou $2^{-l}$
$\Rightarrow$ $A$ akceptuje s pravdepodobnosťou $2^{-2} + 2^{-3} + 2^{-4} =
7/16$, zamietne s $ 2^{-2} + 2^{-2} + 2^{-4}=\frac{9}{16}$.

\paragraph{Definícia} (bounded-error probabilisic polynomial) $$BPP = \{ L \mid
\exists \mbox{pravdepodobnostný polynomiálny algoritmus akceptujúci s chybou}
\leq \varepsilon (0<\varepsilon <\frac{1}{2}\}$$

\paragraph{Veta} $P \subseteq BPP \subseteq PSPACE$.
\paragraph{Dôkaz} $P \subseteq BPP$ -- zrejmé. Nech $L \in BPP$ , nech $A$ je
pravdepodobnostný algoritmus akceptujúci $L$ v polynomiálnom čase $p(x)$ s
chybou $\varepsilon$. Nech $M$ je deterministický Turingov stroj s pamäťou
$p(n)$, ktorý na vstupe $x$:
\begin{itemize}
	\item na jednej z pracovných pások postupne generuje všetky binárne
	postupnosti dĺžky $\leq p(n)$.
	\item po vygenerovaní každdej postupnosti stroj $M$ simuluje výpočet
	algoritmu $A$ na vstupe $x$ s hodnotami $brand$ určenýmmi vygenerovanou
	postupnosťou, pričom existuje ignoruje postupnosti s nevhodným počtom
	volaní $brand$.
	\item $M$ priebežne počíta pravdepodobnosť toho, že $A$ akceptuje $x$.
	\item $M$ akceptuje $x$, ak $A$ akceptuje $x$ s pravdepodobnosťou $\geq
	1 - \varepsilon$, inak $M$ zamietne $x$. Z toho vyplýva, že $L \in
	PSPACE$.
\end{itemize}

\paragraph{Dôsledok} $$P \subseteq \left\{ 
\begin{array}{l}
NP \\
BPP \\
\end{array}
\right\}
\subseteq PSPACE
$$
\paragraph{Poznámka} 
\begin{itemize}
	\item nie je známe, či $P=BPP$ alebo či $BPP=PSPACE$
	\item nie je známy žiaden vzťah $NP$ a $BPP$
\end{itemize}

\paragraph{Veta} (o vylepšovaní) Nech $A$ je pravdepodobnostný algoritmus
akceptujúci $L \subseteq \Sigma^*$ v čase $T(n)$ s chybou $\varepsilon$, $(0 <
\varepsilon < \frac{1}{2})$. Potom $\forall \varepsilon' (0<\varepsilon'<
\varepsilon$ existuje pravdepodobnostný algoritmus $A'$ akceptujúci $L$ v čase
$O(T(n))$ s chybou $\varepsilon'$.

\paragraph{Dôkaz}
Algoritmus $A'$: $A'$ akceptuje (zamietne) vstup $x$, ak spomedzi $m$ náhodne
vybraných výpočtov $A$ na $x$ ($A'$ ich vyberie a simuluje) je počet
akceptujúcich (zamietajúcich) väčší než počet zamietajúcich (akceptujúcich). $m$
je nepárne číslo, t.j. $\frac{1}{2}(4(1-\varepsilon)\varepsilon)^{m/2} \leq
\varepsilon'$. (Také $m$ existuje, lebo $0 < 4(1-\varepsilon)\varepsilon =
4(\frac{1}{2} + \alpha)(\frac{1}{2} - \alpha) = 1 - 4\alpha^2 < 1$ pre $ 0<
\varepsilon < \frac{1}{2}$ a $O < \alpha '\frac{1}{2} - \varepsilon <
\frac{1}{2}$). Pravdepodobnosť chyby $A'$ na $x$ je  $\leq
\frac{1}{2}(4(1-\varepsilon)\varepsilon)^{m/2}$, lebo:
\subparagraph{Prípad 1} $x\in L$ Nech $\varepsilon_x$ je pravdepodobnosť toho,
že náhodne vybraný výpočet $A$ na $x$ je zamietajnúci, teda $O\leq \varepsilon_x
\leq \varepsilon$ a $1-\varepsilon_x$ = pravdepodobnosť toho, že náhodne vybraný
výpočet $A$ na $x$ je akceptujúci.

\paragraph{Príklad} Nech $p$ je pravdepodobnosť toho, že postupnosť 5 náhodne
vybraných výpoštov $A$ na $x$ je tvaru $a,z,z,a,z$ ($a$ -- akceptujúci, $z$ --
zamietajúci):
$$p=(1-\varepsilon_x)\varepsilon_x\varepsilon_x(1-\varepsilon_x)\varepsilon_x =
(1-\varepsilon_x)^2\varepsilon_x^3$$.
$\binom{5}{2}$ = počet postupností dĺžky 5 s dvomi \uv{a} a tromi \uv{z}. Teda
$\binom{5}{2}(1-\varepsilon_x)^2\varepsilon_x^3 $ =  pravdepodobnosť toho, že
postupnosť 5 náhodných výpočtov $A$ na $x$ obsahuje 2 akceptačné a 3 zamietajúce
výpočty. 
$$ \sum_{j=0}^{\lfloor m/2 \rfloor} \binom{m}{j}
(1-\varepsilon_x)^j\varepsilon_x^{m-j}$$
Pravdepodobnosť toho, že posledných $m$ náhodne vybraných výpočtov $A$ na $x$
obsahuje $\leq \lfloor m/2 \rfloor$ akceptačných výpočtov (t.j. obsahuje viac
zamietajúcich než zamietajúcich výpočtov) = pravdepodobnosť toho, že $A'$
zamietne $x\in L$ = pravdepodobnosť chyby $A'$ na $x\in L$.

\paragraph{Prípad 1a} $\varepsilon_x = 0 \Rightarrow \sum_{j=0}^{\lfloor m/2
\rfloor} \binom{m}{j}(1-\varepsilon_x)^j\varepsilon_x^{m-j} = 0 <
\frac{1}{2}(4(1-\varepsilon)\varepsilon)^{m/2}$.

\paragraph{Prípad 1b} $\varepsilon_x > 0 \Rightarrow 1 <
\frac{1-\varepsilon_x}{\varepsilon_x}$ (lebo $O<\varepsilon_x \leq \varepsilon
<\frac{1}{2}$ $\Rightarrow$ $\sum_{j=0}^{\lfloor m/2 \rfloor}
\binom{m}{j}(1-\varepsilon_x)^j\varepsilon_x^{m-j} \leq \sum_{j=0}^{\lfloor m/2
\rfloor}\binom{m}{j}(1-\varepsilon_x)^j\varepsilon_x^{m-j}\underbrace{\left(\frac{1-\varepsilon_x}{\varepsilon_x}\right)^{m/2
- j}}_{>1} = 
\underbrace{\sum_{j=0}^{\lfloor m/2 \rfloor}\binom{m}{j}}_{2^m/2 =
\frac{1}{2}((2^2)^{m/2}}(+-\varepsilon_x)^{m/2}\varepsilon_x^{m/2} =
\frac{1}{2}(2^2(1-\varepsilon_x)\varepsilon_x)^{m/2} \leq
\frac{1}{2}(4(1-\varepsilon)\varepsilon)^{m/2}$

Posledná nerovnosť nie je triviálna:

$(1-\varepsilon) \varepsilon = (1-\varepsilon_x)\varepsilon_x + 
\underbrace{(\varepsilon - \varepsilon_x)}_{\geq 0}
\underbrace{(1-\varepsilon - \varepsilon_x)}_{\geq 0} \geq
(1-\varepsilon_x) \varepsilon_x$

\subparagraph{Príklad 2} $x \notin L$. Dôkaz je podobný prípadu 1.
Pravdepodobnosť chyby $A'$ na $x$ je $\leq
\frac{1}{2}(4(1-\varepsilon!\varepsilon)^{m/2} \underbrace{\leq
\varepsilon'}_{\mbox{výber}~m~\mbox{v}~A'}$ $\Rightarrow$ veta.

\paragraph{Dôsledok} (dôkazu). Ak pre kaťdé $n$ vykoná $A'$ na každodm vstupe
$x$ dĺžky $n$ práve $2n+1$ náhodne vybraných výpočtov $A$ na $x$, potom $A'$
akceptuje $L \cap \Sigma^n$ v čase $O(nT(n))$ s chybou 
$\frac{1}{2}(\underbrace{4(1-\varepsilon)\varepsilon}_{<1})^{n+\frac{1}{2}}$
\end{document}
