\documentclass{article}
\usepackage{slovak}
\usepackage[utf8]{inputenc}
\usepackage{amsmath}
\usepackage{amssymb}
\begin{document}

\paragraph{Definícia} Logický obvod s $n$ vstupmi $x_1, \ldots, x_m$ a
$1$ výstupom je orientovaný acyklický graf taký, že:
\begin{itemize}
	\item v každom vrchole je buď niektore $x_i$ (vrchol bez vstupných hrán)
	alebo $AND$, $OR$ (vrchol s $2$ vstupnými hranami) alebo $NOT$ (vrchol s
	$1$ vstupnou hranou), počet výstupných hrán je neobmedzený.
	\item jeden z vrcholov je výstupný
\end{itemize}

\paragraph{Definícia} \emph {Charakteristické} funkcia pre množinu $D \subseteq
\{ 0,1 \}^n$ je booleovská funkcia $f(x_1,\ldots ,x_n$, taká, že $f(x_1, \ldots,
x_n) = 1 \iff x_1, \ldots,x_n \in D$.

\paragraph{Veta} Nech $L \in BPP$, $L\subseteq \{0,1\}^*$, Potom platia
nasledovné dve tvrdenia:
\begin{enumerate}
	\item Pre jazykk $L$ existuje polynóm $p(x)$ a pravdepodobnostný
	polynomiálny algoritmus $B$ s vlatnosťou:
	$$ \forall n \exists y_n \in \{ 0, 1\}^{p(n)} \mbox{ taký, že }  \forall
	x \in \{0,1\}^n \mbox{ platí }:$$
	výpočet $B$ na $x$ s hodnotami $brand$ určenými reťazcom $y_n$ je
	správny (teda algoritmus $B$ poznajúc $y_n$ neurobí chybu na žiadnom
	vstupe $x \in \{ 0,1 \}^n$.
	\item Pre $L$ existuje polynóm $q(n)$ taký, že $\forall n$ existuje
	logický obvod s $n$ vstupmi, 1 výstupom a najviac $q(n)$ vrcholmi, ktorý
	počíta charakteristickú funkciu pre $L \cap \{0,1\}^n$.
\end{enumerate}

\paragraph{Dôkaz} Veta o vylepšovaní $\implies$ existuje pravdepodobnostný
algoritmus $A$ akceptujúci $L$ v nejakom polynomiálnom čae $s(n)$ s chybou
$\frac{1}{7}$ ( $= \Sigma'$ ).

\paragraph{Dôsledok (dôkazu)} Existuje pravdepodobnostný algoritmus $A'$, ktorý
v čase $O(nS(n))$ akceptuje slová patriace do $L \cap \{0,1\}^n$ s chybou
$\frac{1}{2}(s(1-\frac{1}{7})\frac{1}{7})^{n+\frac{1}{2}} =
\frac{1}{2}\left(\frac{24}{49}\right)^{n+\frac{1}{2}} <
\left(\frac{1}{2}\right)^{n+\frac{3}{2}} \forall n$.

\paragraph{Fakt} Nech $C_1, \ldots, C_r$ sú všetky nesprávne výpočty $A'$ na $x$
s počtom $l_1, \ldots, l_r$ volaní procedúry $brand$. Potom $\sum_{i=1}^{r}
2^{-l_i} = $ pravdepodobnosť chyby  $A'$ na $x$ (lebo $A'$ vykoná $c_i$ s
pravdepodobnosťou $2^{-l_i}$.
\begin{figure}
nesprávny výpočet ($x\notin L$): $C_1, C_2, C_3$
$ l_1 = 2, l_2 = 3, l_3 = 4 $
%pic1
\end{figure}

Upravme $A'$ tak, aby po zistení výsledku (akceptuj/zamietni) vykonal na $x$
dĺžky $n$ ešte toľko volaní $brand$, aby ich celkový počet bol $(2n+1)s(n)$.
Nech $A''$ označuje upravený $A'$.

$\forall x \in \{0,1\}^n: Y_x^n = \{ y \in \{ 0,1 \}^{(2n+1)s(n)} |
\mbox{výpočet A' na $x$ určený reťazcom  $y$ je nesprávny}\}$

$|Y_x^n| = $ počet nesprávnych výpočtov na $A''$  na $x$ je $\sim_{i=1}^{r}
2^{(2n-1)s(n) - l_i} = 2^{(2n+1)s(n)}$ (pravdepodobnosť chyby $A''$ na $x$)
$\leq$ (*) $2^{(2n+1)s(n)}\cdot (\frac{1}{2})^{n+\frac{3}{2})}$ (**).

$$ |Z^n| \geq - \sum_{x\in\{0,1\}^n} |Y_x^n| \geq (**) 2^{(2n+1)s(n)}
(1-2^n(\frac{1}{2}^{n+\frac{3}{2}})$$
$$ = 2^{(2n+1)s(n)(\underbrace{\geq
\frac{1}{2}}{1-(\frac{1}{2}^{\frac{3}{2}}})} \geq 1$$ 
$$ \implies Z^n \neq \emptyset \implies \forall n \exists y_n \implies$$ výpočet
$A''$ na $x$ určený reťazcom $y_n$ je správny $\forall s \in \{0,1\}^n$
$\implies$
(a) platí pre $B = A''$ a $p(n)  = (2n+1)s(n)$.

\paragraph{Dôkaz (2)} Idea: Logický obvod počítajúci charakteristickoú funkciu
pre $l \cap \{0,1\}^n$ zostrojíme tak, aby simuloval výpočty agoritmu $B$ (pre
jednoduchosť deterministický Turingov stroj $M$ simulujúceho $B$) na $x \in
\{0,1\}^n$ určené reťazcom $y_n$.

\begin{figure}[h]
\begin{verbatim}
             |<---n--->|
	|cent|   x     |#|   y_n   | ... 
               ^
	       |
	       q 	M = (\Sigma, Q, q_, \delta, F)                
\end{verbatim}
\caption{T-Stroj $M$}
\end{figure}
Štruktúra obvodu:
(chýba obrázok)
%\begin{figure}
%        \vdots          \vdots
%    ... D_{i-1,j}       D_{i,j}        D_{i+1,j}
%            |     \
%            v      \--\
%    ... D_{i-1,j+1}     D_{i,j+1}      D_{i+1,j+1}
%    (z ľubovoľného stvroca na i-tej vrstve ide sipka do kazdeho stvorca na
%    (i+1)-vej.
%\end{figure}

$\forall i,j D_{i,j}$ má na vstupe binárne kódované nasledovné informácie pre
$j$-ty krok výpočtu stroja $M$:
\begin{itemize}
	\item obsaho $i$-teho políčka  ($\lceil \log |\Sigma| \rceil$ bitov).
	\item či je $i$-te políčko čítané (1 bit)
	\item stav $q \in Q$ (ak je $i$-te políčko čítané ( $\lceil \log
	|Q|\rceil$ bitov).
	\item pohyb hlavy  (ak je $i$-te políčko čítané) 1/0/-1 (2 bity)
\end{itemize}

Bloky $D_{i,j}$ majú $3k$ bitov na vstupe a $k$  bitov na výstupe. Každý blok
$D_{i,j}$ treba nahradiť pomocou $k$ logických obvodov s $3k$ vstupmi a $1$
výstupom. Výsledný logický obvod  počíta charakteristickú funkciu pre $L \cap
\{0,1\}^n$ a má polynomiálne veľa vrcholov.

\begin{verbatim}
procedure Freivalds (A,B,C,r) {
// overí rovnosť AB=C s pravdepodobnosťou omylu $2^{-r}$
// A,B,C sú matice rádu $n*n$
náhodne vyber bin. bektory z_1, z_2, \ldots z_r dĺžky n
if (z_i, A)B = z_i C \forall i then return "yes"  (s pravdepodobnosťou omylu
\leq 2^{-r})
else return "no" (bez omylu)
}
\end{verbatim}

Časová zložitosť $O(rn^2)$.

\end{document}
